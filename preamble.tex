%%%%%%%%%%%%%%%%%%%%%%%%%%%%%%%%%%%%%%%%%%%%%%%%%%%%%%%%%%%%%%%%%%%%%%%%%%%%%%%
%                                Basic Packages                               %
%%%%%%%%%%%%%%%%%%%%%%%%%%%%%%%%%%%%%%%%%%%%%%%%%%%%%%%%%%%%%%%%%%%%%%%%%%%%%%%

% Gives us multiple colors.
\usepackage[usenames,dvipsnames,pdftex]{xcolor}
% Lets us style link colors.
\usepackage{hyperref}
% Lets us import images and graphics.
\usepackage{graphicx}
% Lets us use figures in floating environments.
\usepackage{float}
% Lets us create multiple columns.
\usepackage{multicol}
% Gives us better math syntax.
\usepackage{amsmath,amsfonts,mathtools,amsthm,amssymb}
% Lets us strikethrough text.
\usepackage{cancel}
% Lets us edit the caption of a figure.
\usepackage{caption}
% Lets us import pdf directly in our tex code.
\usepackage{pdfpages}
% Lets us do algorithm stuff.
\usepackage[ruled,vlined,linesnumbered]{algorithm2e}

\usepackage{geometry}[margin = 1cm]
\usepackage[T1]{fontenc}
\usepackage{lmodern}

\def\class{article}


%%%%%%%%%%%%%%%%%%%%%%%%%%%%%%%%%%%%%%%%%%%%%%%%%%%%%%%%%%%%%%%%%%%%%%%%%%%%%%%
%                                Basic Settings                               %
%%%%%%%%%%%%%%%%%%%%%%%%%%%%%%%%%%%%%%%%%%%%%%%%%%%%%%%%%%%%%%%%%%%%%%%%%%%%%%%

%%%%%%%%%%%%%
%  Symbols  %
%%%%%%%%%%%%%

\let\implies\Rightarrow
\let\impliedby\Leftarrow
\let\iff\Leftrightarrow
\let\epsilon\varepsilon

%%%%%%%%%%%%
%  Tables  %
%%%%%%%%%%%%

\setlength{\tabcolsep}{5pt}
\renewcommand\arraystretch{1.5}

%%%%%%%%%%%%%%
%  SI Unitx  %
%%%%%%%%%%%%%%

\usepackage{siunitx}
\sisetup{locale = FR}

%%%%%%%%%%
%  TikZ  %
%%%%%%%%%%

\usepackage[framemethod=TikZ]{mdframed}
\usepackage{tikz}
\usepackage{tikz-cd}
\usepackage{tikzsymbols}

\usetikzlibrary{intersections, angles, quotes, calc, positioning}
\usetikzlibrary{arrows.meta}

\tikzset{
  force/.style={thick, {Circle[length=2pt]}-stealth, shorten <=-1pt}
}

%%%%%%%%%%%%%%%
%  PGF Plots  %
%%%%%%%%%%%%%%%

\usepackage{pgfplots}
\pgfplotsset{compat=1.13}

%%%%%%%%%%%%%%%%%%%%%%%
%  Center Title Page  %
%%%%%%%%%%%%%%%%%%%%%%%

\usepackage{titling}
\renewcommand\maketitlehooka{\null\mbox{}\vfill}
\renewcommand\maketitlehookd{\vfill\null}

%%%%%%%%%%%%%%%%%%%%%%%%%%%%%%%%%%%%%%%%%%%%%%%%%%%%%%%
%  Create a grey background in the middle of the PDF  %
%%%%%%%%%%%%%%%%%%%%%%%%%%%%%%%%%%%%%%%%%%%%%%%%%%%%%%%

\usepackage{eso-pic}
\newcommand\definegraybackground{
  \definecolor{reallylightgray}{HTML}{FAFAFA}
  \AddToShipoutPicture{
    \ifthenelse{\isodd{\thepage}}{
      \AtPageLowerLeft{
        \put(\LenToUnit{\dimexpr\paperwidth-222pt},0){
          \color{reallylightgray}\rule{222pt}{297mm}
        }
      }
    }
    {
      \AtPageLowerLeft{
        \color{reallylightgray}\rule{222pt}{297mm}
      }
    }
  }
}

%%%%%%%%%%%%%%%%%%%%%%%%
%  Modify Links Color  %
%%%%%%%%%%%%%%%%%%%%%%%%

\hypersetup{
  % Enable highlighting links.
  colorlinks,
  % Change the color of links to blue.
  linkcolor=blue,
  % Change the color of citations to black.
  citecolor={black},
  % Change the color of url's to blue with some black.
  urlcolor={blue!80!black}
}

%%%%%%%%%%%%%%%%%%
% Fix WrapFigure %
%%%%%%%%%%%%%%%%%%

\newcommand{\wrapfill}{\par\ifnum\value{WF@wrappedlines}>0
    \parskip=0pt
    \addtocounter{WF@wrappedlines}{-1}%
    \null\vspace{\arabic{WF@wrappedlines}\baselineskip}%
    \WFclear
\fi}

%%%%%%%%%%%%%%%%%
% Multi Columns %
%%%%%%%%%%%%%%%%%

\let\multicolmulticols\multicols
\let\endmulticolmulticols\endmulticols

\RenewDocumentEnvironment{multicols}{mO{}}
{%
  \ifnum#1=1
    #2%
  \else % More than 1 column
    \multicolmulticols{#1}[#2]
  \fi
}
{%
  \ifnum#1=1
\else % More than 1 column
  \endmulticolmulticols
\fi
}

\newlength{\thickarrayrulewidth}
\setlength{\thickarrayrulewidth}{5\arrayrulewidth}


%%%%%%%%%%%%%%%%%%%%%%%%%%%%%%%%%%%%%%%%%%%%%%%%%%%%%%%%%%%%%%%%%%%%%%%%%%%%%%%
%                           Specific Commands                          %
%%%%%%%%%%%%%%%%%%%%%%%%%%%%%%%%%%%%%%%%%%%%%%%%%%%%%%%%%%%%%%%%%%%%%%%%%%%%%%%

% numbering equations
\renewcommand{\theequation}{\thechapter.\arabic{equation}}
\counterwithin{equation}{chapter}

% shortcuts
\def\R{\mathbb{R}}
\def\Q{\mathbb{Q}}
\def\N{\mathbb{N}}
\def\C{\mathbb{C}}
\def\Z{\mathbb{Z}}

\def\a{\alpha}
\def\b{\beta}
\def\c{\gamma}
\def\d{\delta}
\def\e{\epsilon}
\def\h{\theta}
\def\l{\lambda}
\def\w{\omega}

\def\bu{\mathbf{u}}
\def\bv{\mathbf{v}}
\def\bw{\mathbf{w}}
\def\bx{\mathbf{x}}
\def\bf{\mathbf}

\def\la{\langle}
\def\ra{\rangle}

\def\st{\text{ s.t. }}
\def\ust{Please don't go to UST}

\def\iff{\Longleftrightarrow}
\def\to{\rightarrow}
\def\To{\Rightarrow}
\def\inj{\hookrightarrow}
\def\surj{\twoheadrightarrow}
\def\up{\uparrow}
\def\down{\downarrow}

\def\x{\times}
\def\<{\langle}
\def\>{\rangle}
\def\oo{\infty}
\def\normal{\triangleleft}
\def\ds{\displaystyle}

\newcommand{\mat}[1]{\begin{pmatrix}#1 \\ \end{pmatrix}}
\newcommand{\case}[2][lllllllllllllllllllllllllllllllllllll]{\left\{\begin{array}{#1}#2 \\ \end{array}\right.}
\newcommand{\Eq}[1]{\begin{align}#1\end{align}} %Equations with numberings%
\newcommand{\Eqn}[1]{\begin{align*}#1\end{align*}}  % Equations without numberings%
\newcommand{\floor}[1]{\lfloor#1\rfloor}
\newcommand{\abs}[1]{\bigg| #1 \bigg|}

\DeclareMathOperator{\Span}{Span}
\DeclareMathOperator{\Ker}{Ker}
\DeclareMathOperator{\Image}{Im}
\DeclareMathOperator{\Nul}{Nul}
\DeclareMathOperator{\Col}{Col}
\DeclareMathOperator{\Rank}{Rank}
\DeclareMathOperator{\Tr}{Tr}
\DeclareMathOperator{\Proj}{Proj}
\DeclareMathOperator{\diag}{diag}

%%%%%%%%%%%%%%%%%%%%%%%%%%%
%  Initiate New Counters  %
%%%%%%%%%%%%%%%%%%%%%%%%%%%

\newcounter{lecturecounter}

%%%%%%%%%%%%%%%%%%%%%%%%%%
%  Helpful New Commands  %
%%%%%%%%%%%%%%%%%%%%%%%%%%

\makeatletter

\newcommand\resetcounters{
  % Reset the counters for subsection, subsubsection and the definition
  % all the custom environments.
  \setcounter{subsection}{0}
  \setcounter{subsubsection}{0}
  \setcounter{paragraph}{0}
  \setcounter{subparagraph}{0}
  \setcounter{theorem}{0}
  \setcounter{claim}{0}
  \setcounter{corollary}{0}
  \setcounter{lemma}{0}
  \setcounter{exercise}{0}

  \@ifclasswith\class{nocolor}{
    \setcounter{definition}{0}
  }{}
}

%%%%%%%%%%%%%%%%%%%%%
%  Lecture Command  %
%%%%%%%%%%%%%%%%%%%%%

\usepackage{xifthen}

% EXAMPLE:
% 1. \lesson{Oct 17 2022 Mon (08:46:48)}{Lecture Title}
% 2. \lesson[4]{Oct 17 2022 Mon (08:46:48)}{Lecture Title}
% 3. \lesson{Oct 17 2022 Mon (08:46:48)}{}
% 4. \lesson[4]{Oct 17 2022 Mon (08:46:48)}{}
% Parameters:
% 1. (Optional) Lesson number.
% 2. Time and date of lecture.
% 3. Lecture Title.
\def\@lesson{}
\newcommand\lesson[3][\arabic{lecturecounter}]{
  % Add 1 to the lecture counter.
  \addtocounter{lecturecounter}{1}

  % Set the section number to the lecture counter.
  %\setcounter{section}{#1}
  %\renewcommand\thesubsection{#1.\arabic{subsection}}

  % Reset the counters.
  %\resetcounters

  % Check if user passed the lecture title or not.
  \ifthenelse{\isempty{#3}}{
    \def\@lesson{Lecture \arabic{lecturecounter}}
  }{
    \def\@lesson{Lecture \arabic{lecturecounter}: #3}
  }

  % Display the information like the following:
  %                                                  Oct 17 2022 Mon (08:49:10)
  % ---------------------------------------------------------------------------
  % Lecture 1: Lecture Title
  \hfill\small{#2}
  \hrule
  \vspace*{-0.3cm}
  \section*{\@lesson}
  \addcontentsline{toc}{section}{\@lesson}
}

%%%%%%%%%%%%%%%%%%%%
%  Import Figures  %
%%%%%%%%%%%%%%%%%%%%

\usepackage{import}
\pdfminorversion=7

% EXAMPLE:
% 1. \incfig{limit-graph}
% 2. \incfig[0.4]{limit-graph}
% Parameters:
% 1. The figure name. It should be located in figures/NAME.tex_pdf.
% 2. (Optional) The width of the figure. Example: 0.5, 0.35.
\newcommand\incfig[2][1]{%
  \def\svgwidth{#1\columnwidth}
  \import{./figures/}{#2.pdf_tex}
}

\begingroup\expandafter\expandafter\expandafter\endgroup
\expandafter\ifx\csname pdfsuppresswarningpagegroup\endcsname\relax
\else
  \pdfsuppresswarningpagegroup=1\relax
\fi

%%%%%%%%%%%%%%%%%
% Fancy Headers %
%%%%%%%%%%%%%%%%%

\usepackage{fancyhdr}

% Force a new page.
\newcommand\forcenewpage{\clearpage\mbox{~}\clearpage\newpage}

% This command makes it easier to manage my headers and footers.
\newcommand\createintro{
  % Use roman page numbers (e.g. i, v, vi, x, ...)
  \pagenumbering{roman}

  % Display the page style.
  \maketitle
  % Make the title pagestyle empty, meaning no fancy headers and footers.
  \thispagestyle{empty}
  % Create a newpage.
  \newpage

  % Input the intro.tex page if it exists.
  \IfFileExists{intro.tex}{ % If the intro.tex file exists.
    % Input the intro.tex file.
    Lecture notes from 2025 Fall MATH3312: Numerical Analysis, given by Prof. Peng Zhichao at the Hong Kong University of Science and Technology in the academic year Fall 2025. This
course covers computer arithmetic, root-finding, interpolation, numerical calculus and solving linear systems. 



\textit{Disclaimer:} This document will inevitably contain some mistakes -- both simple typos and legitimate errors. Keep in mind that these are the notes of an undergraduate student in the process of learning the material himself, so take what you read with a grain of salt. If you find mistakes and feel like telling me, I will be grateful and happy to hear from you, even for the most trivial of errors. You can reach me by email, in English at \href{mailto:mycheungai@conenct.ust.hk}{mycheungai@conenct.ust.hk}.


    % Make the pagestyle fancy for the intro.tex page.
    \pagestyle{fancy}

    % Remove the line for the header.
    \renewcommand\headrulewidth{0pt}

    % Remove all header stuff.
    \fancyhead{}

  }{ % If the intro.tex file doesn't exist.
    % Force a \newpageage.
    \forcenewpage
  }

  % Create a new page.
  \newpage

  % Remove the center stuff we did above, and replace it with just the page
  % number, which is still in roman numerals.
  \fancyfoot[C]{\thepage}
  % Add the table of contents.
  \tableofcontents
  % Force a new page.
  \forcenewpage

  % Move the page numberings back to arabic, from roman numerals.
  \pagenumbering{arabic}
  % Set the page number to 1.
  \setcounter{page}{1}

% clear existing header/footer
\fancyhf{}

% chapter on top-left, section on top-right
\fancyhead[L]{\leftmark}
\fancyhead[R]{\rightmark}

% page number at bottom center
\fancyfoot[C]{\thepage}

% rule under header
\renewcommand{\headrulewidth}{0.4pt}
\renewcommand{\footrulewidth}{0pt}

}

\makeatother


%%%%%%%%%%%%%%%%%%%%%%%%%%%%%%%%%%%%%%%%%%%%%%%%%%%%%%%%%%%%%%%%%%%%%%%%%%%%%%%
%                               Custom Commands                               %
%%%%%%%%%%%%%%%%%%%%%%%%%%%%%%%%%%%%%%%%%%%%%%%%%%%%%%%%%%%%%%%%%%%%%%%%%%%%%%%

%%%%%%%%%%%%
%  Circle  %
%%%%%%%%%%%%

\newcommand*\circled[1]{\tikz[baseline=(char.base)]{
  \node[shape=circle,draw,inner sep=1pt] (char) {#1};}
}

%%%%%%%%%%%%%%%%%%%
%  Todo Commands  %
%%%%%%%%%%%%%%%%%%%

\usepackage{xargs}
\usepackage[colorinlistoftodos]{todonotes}

\makeatletter

\@ifclasswith\class{working}{
  \newcommandx\unsure[2][1=]{\todo[linecolor=red,backgroundcolor=red!25,bordercolor=red,#1]{#2}}
  \newcommandx\change[2][1=]{\todo[linecolor=blue,backgroundcolor=blue!25,bordercolor=blue,#1]{#2}}
  \newcommandx\info[2][1=]{\todo[linecolor=OliveGreen,backgroundcolor=OliveGreen!25,bordercolor=OliveGreen,#1]{#2}}
  \newcommandx\improvement[2][1=]{\todo[linecolor=Plum,backgroundcolor=Plum!25,bordercolor=Plum,#1]{#2}}

  \newcommand\listnotes{
    \newpage
    \listoftodos[Notes]
  }
}{
  \newcommandx\unsure[2][1=]{}
  \newcommandx\change[2][1=]{}
  \newcommandx\info[2][1=]{}
  \newcommandx\improvement[2][1=]{}

  \newcommand\listnotes{}
}

\makeatother

%%%%%%%%%%%%%
%  Correct  %
%%%%%%%%%%%%%

% EXAMPLE:
% 1. \correct{INCORRECT}{CORRECT}
% Parameters:
% 1. The incorrect statement.
% 2. The correct statement.
\definecolor{correct}{HTML}{009900}
\newcommand\correct[2]{{\color{red}{#1 }}\ensuremath{\to}{\color{correct}{ #2}}}


%%%%%%%%%%%%%%%%%%%%%%%%%%%%%%%%%%%%%%%%%%%%%%%%%%%%%%%%%%%%%%%%%%%%%%%%%%%%%%%
%                                 Environments                                %
%%%%%%%%%%%%%%%%%%%%%%%%%%%%%%%%%%%%%%%%%%%%%%%%%%%%%%%%%%%%%%%%%%%%%%%%%%%%%%%

\usepackage{varwidth}
\usepackage{thmtools}
\usepackage[most,many,breakable]{tcolorbox}

\tcbuselibrary{theorems,skins,hooks}
\usetikzlibrary{arrows,calc,shadows.blur}

%%%%%%%%%%%%%%%%%%%
%  Define Colors  %
%%%%%%%%%%%%%%%%%%%

\definecolor{definition}{HTML}{A37800}
\definecolor{theorem}{HTML}{C94503}
\definecolor{example}{HTML}{537817}
\definecolor{nonexample}{HTML}{B80006}
\definecolor{exercise}{HTML}{3A2A51}
\definecolor{question}{HTML}{10547E}

\colorlet{solution}{question}
\colorlet{claim}{theorem}
\colorlet{corollary}{theorem}
\colorlet{prop}{theorem}
\colorlet{lemma}{theorem}
\colorlet{proof}{theorem}

%%%%%%%%%%%%%%%%%%%%%%%%%%%%%%%%%%%%%%%%%%%%%%%%%%%%%%%%%
%  Create Environments Styles Based on Given Parameter  %
%%%%%%%%%%%%%%%%%%%%%%%%%%%%%%%%%%%%%%%%%%%%%%%%%%%%%%%%%

\mdfsetup{skipabove=1em,skipbelow=0em}

%%%%%%%%%%%%%%%%%%%%%%
%  Helpful Commands  %
%%%%%%%%%%%%%%%%%%%%%%

% EXAMPLE:
% 1. \createnewtheoremstyle{thmdefinitionbox}{}{}
% 2. \createnewtheoremstyle{thmtheorembox}{}{}
% 3. \createnewtheoremstyle{thmproofbox}{qed=\qedsymbol}{
%       rightline=false, topline=false, bottomline=false
%    }
% Parameters:
% 1. Theorem name.
% 2. Any extra parameters to pass directly to declaretheoremstyle.
% 3. Any extra parameters to pass directly to mdframed.

\newcommand\createexamplestyle[4]{
  \declaretheoremstyle[
  headfont=\bfseries\sffamily\color{#2}, bodyfont=\normalfont, #3,
  mdframed={
    linewidth = 2pt,
    linecolor=#2, backgroundcolor=#2!5
    #4,
  },
  ]{#1}
}

% EXAMPLE:
% 1. \createnewcoloredtheoremstyle{thmdefinitionbox}{definition}{}{}
% 2. \createnewcoloredtheoremstyle{thmexamplebox}{example}{}{
%       rightline=true, leftline=true, topline=true, bottomline=true
%     }
% 3. \createnewcoloredtheoremstyle{thmproofbox}{proof}{qed=\qedsymbol}{backgroundcolor=white}
% Parameters:
% 1. Theorem name.
% 2. Color of theorem.
% 3. Any extra parameters to pass directly to declaretheoremstyle.
% 4. Any extra parameters to pass directly to mdframed.
\newcommand\createtheoremstyle[4]{
  \declaretheoremstyle[
  headfont=\bfseries\sffamily\color{#2}, bodyfont=\normalfont, #3,
  mdframed={
    linewidth=2pt,
    rightline=false, leftline=true, topline=false, bottomline=false,
    linecolor=#2, backgroundcolor=#2!5, #4,
  },
  ]{#1}
}

%%%%%%%%%%%%%%%%%%%%%%%%%%%%%%%%%%%
%  Create the Environment Styles  %
%%%%%%%%%%%%%%%%%%%%%%%%%%%%%%%%%%%

\makeatletter
\@ifclasswith\class{nocolor}{
  % Environments without color.
  \createtheoremstyle{thmdefinitionbox}{black}{}{backgroundcolor=white}
  \createtheoremstyle{thmtheorembox}{black}{}{backgroundcolor=white}
  \createexamplestyle{thmexamplebox}{black}{}{backgroundcolor=white}
  \createexamplestyle{thmnonexamplebox}{black}{}{backgroundcolor=white}
  \createtheoremstyle{thmclaimbox}{black}{}{backgroundcolor=white}
  \createtheoremstyle{thmcorollarybox}{black}{}{backgroundcolor=white}
  \createtheoremstyle{thmpropbox}{black}{}{backgroundcolor=white}
  \createtheoremstyle{thmlemmabox}{black}{}{backgroundcolor=white}
  \createtheoremstyle{thmexercisebox}{black}{}{backgroundcolor=white}
  \createexamplestyle{thmquestionbox}{black}{}{backgroundcolor=white}
  \createexamplestyle{thmsolutionbox}{black}{}{backgroundcolor=white}

  \createtheoremstyle{thmproofbox}{black}{qed=\qedsymbol}{backgroundcolor=white}
  \createtheoremstyle{thmexplanationbox}{black}{\qedsymbol}{backgroundcolor=white}
  \declaretheoremstyle[
  headfont=\bfseries\sffamily\color{black},
  bodyfont=\normalfont,
  headpunct={},
  postheadspace=0pt,
  mdframed={
    linewidth=2pt,
    linecolor=black, backgroundcolor=white,
  },
]{thmnamelessbox}
}{
  % Environments with color.
  \createtheoremstyle{thmdefinitionbox}{definition}{}{}
  \createtheoremstyle{thmtheorembox}{theorem}{}{}
  \createexamplestyle{thmexamplebox}{example}{}{}
  \createexamplestyle{thmnonexamplebox}{nonexample}{}{}
  \createtheoremstyle{thmclaimbox}{claim}{}{}
  \createtheoremstyle{thmcorollarybox}{corollary}{}{}
  \createtheoremstyle{thmpropbox}{prop}{}{}
  \createtheoremstyle{thmlemmabox}{lemma}{}{}
  \createtheoremstyle{thmexercisebox}{exercise}{}{}
  \createexamplestyle{thmquestionbox}{question}{}{}
  \createexamplestyle{thmsolutionbox}{solution}{}{}
  \createtheoremstyle{thmproofbox}{proof}{qed=\qedsymbol}{backgroundcolor=white}
  \createtheoremstyle{thmexplanationbox}{example}{qed=\qedsymbol}{backgroundcolor=white}
  \declaretheoremstyle[
  headfont=\bfseries\sffamily\color{theorem},
  bodyfont=\normalfont,
  headpunct={},
  postheadspace=0pt,
  mdframed={
    linewidth=2pt,
    linecolor=theorem, backgroundcolor=theorem!5,
  },
]{thmnamelessbox}
}
\makeatother

%%%%%%%%%%%%%%%%%%%%%%%%%%%%%
%  Create the Environments  %
%%%%%%%%%%%%%%%%%%%%%%%%%%%%%

% --- Shared theorem counter (chapter.theorem) ---
% Make sure your document class actually HAS chapters (e.g., book/report).
% If you are on article, switch to section or define a chapter counter.


% Base environment: Theorem (drives the shared counter)
\declaretheorem[
  numberwithin=chapter,
  style=thmtheorembox,
  name=Theorem
]{theorem}

% Everything else shares the theorem counter:
\declaretheorem[sibling=theorem, style=thmdefinitionbox, name=Definition]{definition}
\declaretheorem[sibling=theorem, style=thmexamplebox,    name=Example]{example}
\declaretheorem[sibling=theorem, style=thmnonexamplebox,    name=Non-Example]{nonexample}
\declaretheorem[sibling=theorem, style=thmclaimbox,      name=Claim]{claim}
\declaretheorem[sibling=theorem, style=thmcorollarybox,  name=Corollary]{corollary}
\declaretheorem[sibling=theorem, style=thmpropbox,       name=Proposition]{prop}
\declaretheorem[sibling=theorem, style=thmlemmabox,      name=Lemma]{lemma}
\declaretheorem[sibling=theorem, style=thmexercisebox,   name=Exercise]{exercise}
\declaretheorem[sibling=theorem, style=thmquestionbox,   name=Question]{question}
\declaretheorem[numbered=no, style=thmsolutionbox,   name=Solution]{solution}
% Unnumbered environments (unchanged)
\declaretheorem[numbered=no, style=thmproofbox,       name=Proof]{replacementproof}
\declaretheorem[numbered=no, style=thmexplanationbox, name=Explanation]{expl}
\declaretheorem[
  numbered=no,
  name={},
  style=thmnamelessbox
]{nameless}






%%%%%%%%%%%%%%%%%%%%%%%%%%%%
%  Edit Proof Environment  %
%%%%%%%%%%%%%%%%%%%%%%%%%%%%

\renewenvironment{proof}[1][\proofname]{\vspace{-10pt}\begin{replacementproof}}{\end{replacementproof}}
\newenvironment{explanation}[1][\proofname]{\vspace{-10pt}\begin{expl}}{\end{expl}}


\newtheorem*{note}{Note}
\newtheorem*{notation}{Notation}
\newtheorem*{remark}{Remark}