\chapter{Computer Arithmetic}

\lesson[1]{01/09/2025}{Introduction, Floating Number and Round-Off Error}

\section{Floating Number}

\subsection{Decimal and Binary Numbers}
In daily life, numbers are usually represented in decimals, with allowed digits \(\{0, 1, \cdots, 9\}\). However, computer processes information in binary bits, with allowed bits \(\{0, 1\}\).

\begin{example}{Binary $\to$ Decimal}
\[
11.101_2 \;=\; 1\cdot 2^1 + 1\cdot 2^0 + 1\cdot 2^{-1} + 0\cdot 2^{-2} + 1\cdot 2^{-3}
= 3.625.
\]
\end{example}

\subsection{Floating Number Representation}
In a binary floating-point system, numbers are represented as
\[
\mathbb{F}=\bigl\{\pm\,0.1d_2\cdots d_{t+1}\times 2^{m}\bigr\}\cup\{0\},
\]
where $d_2,\dots,d_{t+1}\in\{0,1\}$ and $m \in [L, U]$.
\begin{itemize}
    \item \(t\) is precision
    \item \(m\) is exponent
\end{itemize}

\paragraph{IEEE formats.}
\begin{itemize}
  \item \textbf{Single precision (32-bit):} $t=23$, $L=-126$, $U=127=2^7-1$ i.e. 1 bit for sign, 8 bits for exponent, 23 bits for precision.
  \item \textbf{Double precision (64-bit):} $t=52$, $L=-1022$, $U=1023=2^{10}-1$.
\end{itemize}

\subsection{Rounding}
A real number $x\in\mathbb{R}$ is represented by
\[
x \approx \mathrm{fl}(x) \in \mathbb{F}
\]
Typical rounding choices:
\begin{itemize}
  \item Round up / round down
  \item \textbf{Round to nearest (ties to even)} \;-- IEEE 754 default
\end{itemize}
Thus an \emph{inherent} error exists: $x\neq \mathrm{fl}(x)$.

\section{Round off Error}

\paragraph{Machine precision (unit roundoff).}
\[\varepsilon_{\text{machine}}=2^{-(t+1)}.\]
\begin{itemize}
  \item Single: $2^{-24}\sim O(10^{-8})$.
  \item Double: $2^{-53}\sim O(10^{-16})$.
\end{itemize}

\begin{example}[Round-off in trigonometric function]
    $\sin(5\pi)=0$, yet a typical double-precision routine yields a value on the order of $10^{-16}$ (e.g.\ $6.12\times 10^{-16}$ using \verb|julia|).
\end{example}

\subsection{Cancellation in Sums}

Adding (or subtracting) small numbers with big numbers would remove the precision in small numbers.

\begin{example}[Small + Big goes wrong]
\[
x=0.23371258\times 10^{-4},\quad
y=0.33678429\times 10^{2},\quad
z=-0.33677811\times 10^{2}.
\]
\textbf{Exact:} \; $x+y+z = 0.641371258\times 10^{-3}$ (9 significant digits).

\medskip
\noindent\textbf{Add Small + Big first (worse):}
\[
\bigl(\mathrm{fl}(x)+\mathrm{fl}(y)\bigr)+\mathrm{fl}(z)=0.64100000\times 10^{-3}.
\]

\noindent\textbf{Add Big + Big first (better):}
\[
\mathrm{fl}(x)+\bigl(\mathrm{fl}(y)+\mathrm{fl}(z)\bigr)=0.64137126\times 10^{-3}.
\]
\end{example}

\subsection{Cancellation in Differences}
The difference in large numbers is also unstable.
\begin{example}
Consider
\[
\sqrt{1+10^{-16}}-1.
\]
In floating arithmetic, 
\[
\mathrm{fl}(\sqrt{1+10^{-16}})-\mathrm{fl}(1)=0
\]
due to catastrophic cancellation.

\paragraph{Algebraic reformulation (stable).}
\[
\sqrt{1+10^{-16}}-1=\frac{10^{-8}}{\sqrt{1+10^{-16}}+1}.
\]
Then in floating arithmetic,
\[
\mathrm{fl}\!\left(\frac{10^{-8}}{\sqrt{1+10^{-16}}+1}\right)\approx 5\times 10^{-9}.
\]

\paragraph{Code demo (Julia-like).}
\begin{itemize}
  \item \verb|sqrt(1+1e-16) - 1|
  \item \verb|1e-8 / (sqrt(1+1e-16) + 1)|
\end{itemize}
\end{example}

\subsection{Division by Small Numbers}
\begin{example}
\begin{align*}
\mathrm{fl}(10^{-15}) + 1 - 1 &\approx 1.11\times 10^{-15}\quad\text{(very small absolute error)},\\
\text{Exact:}\quad \frac{1+10^{-15}-1}{10^{-15}} &= 1,\\
\text{Floating:}\quad \frac{\mathrm{fl}(10^{-15})+1-1}{\mathrm{fl}(10^{-15})} &\approx 1.11 \quad\text{(huge relative error $\sim 11\%$)}.
\end{align*}

\paragraph{Code demo (Julia-like).}
\begin{itemize}
  \item \verb|(1+1e-15) - 1|
  \item \verb|((1+1e-15)-1)/(1e-15)|
\end{itemize}
\end{example}

\paragraph{A Real-World Cautionary Tale}
Accumulated round-off can snowball over many steps. On June 4, 1996, the ESA \emph{Ariane 5} rocket failed due to mishandling of floating-point conversions / round-off in software.

\section*{Summary}
\begin{itemize}
  \item Converting between binary and decimal
  \item Understanding round-off error and unit roundoff
  \item Typical pitfalls: (i) small+big sums, (ii) subtraction of close numbers, (iii) division by small numbers
\end{itemize}

\paragraph{References.}
\begin{itemize}
  \item Jeffrey R.\ Chasnov, \emph{Numerical Methods}.
  \item Pingwen Zhang \& Tiejun Li, \emph{Numerical Analysis} (Chinese).
\end{itemize}

